\section{Linux e exemplos de utilização} \label{section: linux e exemplos}

Linux é um sistema operativo OpenSource, criado por Linus Torvalds em quando ainda era estudante de ciência da computação na Universidade de Helsinque, Finlândia, e começou a trabalhar no projeto Linux como
um esforço pessoal.

O nome Linux é uma combinação de seu primeiro nome, Linus, e Unix, o sistema operativo que inspirou seus projetos. Na época, a maioria dos sistemas operativos eram proprietários e caros então Torvalds
decidiu criar um sistema operacional que estava disponível gratuitamente para qualquer pessoa.

As primeiras versões do Linux eram usadas principalmente por entusiastas da tecnologia e desenvolvedores de software, mas com o tempo ele cresceu em popularidade e é usado em diversos ambientes. O Linux é 
considerado um dos sistemas operativos mais estáveis, seguros e confiáveis e é amplamente utilizado em servidores, supercomputadores e ambientes corporativos. 
Com o crescimento na sua popularidade e com o desenvolvimento continuo da comunidade hoje Linux tem como principais distribuições: \textbf{Ubuntu, Fedora, Arch, Plasma, KDE, Mint, Manjaro.}

Linux atualmente está presente em diversos ambientes como os quais:

\begin{enumerate}
    \item \textbf{Servidores e Data Centers}\\
    O Linux é amplamente reconhecido pelo seu domínio no mercado de servidores e centros de dados, destacando-se pela sua estabilidade e confiabilidade. É frequentemente utilizado para operar redes de dados
    e data centers.
    Muitos dos equipamentos constituintes dos servidores e data centers, tais como os routers, funcionam com versões personalizadas e simplificadas do sistema operativo Linux.

    \item \textbf{Supercomputadores}\\
    A capacidade do Linux para escalar eficazmente para milhares de núcleos de processamento e a sua flexibilidade para otimização em tarefas de alto desempenho são essenciais para o seu uso em supercomputadores. 
    De facto, o Linux é o sistema operativo preferencial para a maioria dos supercomputadores, evidenciando a sua eficiência em ambientes de computação intensiva.

    \item \textbf{Dispositivos IoT}\\
    No mundo da Internet das Coisas (IoT), o Linux é ideal devido ao seu tamanho reduzido e capacidade de ser adaptado para se ajustar a hardware específico. 
    Desde eletrodomésticos inteligentes até sistemas avançados de controlo industrial e veículos autónomos, o Linux serve como uma base fiável para muitos dispositivos inovadores.
    
    \item \textbf{Desktops e Uso Diário}\\
    Apesar de ser menos popular que o Windows ou MacOS em desktops, o Linux tem vindo a observar um aumento constante na aceitação por parte dos utilizadores. Esta tendência deve-se à sua crescente biblioteca
    de programas de software baseados em Linux e um foco na expansão da oferta de interfaces de utilizador mais amigáveis para desktop.
    O sistema operativo Linux é também o sistema base de outros sistemas operativos que utilizamos no nosso dia a dia como o Sistema operativo Android e Chromebook

    \item \textbf{Educação e Governo}\\
    O baixo custo e a capacidade de personalização do Linux são altamente valorizados por instituições educativas e governamentais. Mundialmente, existem vários governos que têm implementado o Linux de forma
    extensiva nas operações governamentais e sistemas educacionais, aproveitando estas vantagens.
    
\end{enumerate}