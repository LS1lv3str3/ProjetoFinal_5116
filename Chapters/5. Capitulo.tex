\section{Licenciamento \textit{open source}} \label{section: licenciamento}
O licenciamento open source é um aspeto crucial no desenvolvimento de software, servindo como uma fundação para a inovação e colaboração tecnológica.
As licenças open source são rigorosamente aprovadas pela \textit{Open Source Initiative}(OSI) e asseguram que qualquer software sob estas licenças possa ser livremente utilizado, modificado e redistribuído.
Este sistema de licenciamento é vital para manter a integridade e a filosofia da partilha e colaboração que são centrais para a comunidade open source.


\textbf{Definição e Aprovação de Licenças Open Source}
Segundo a OSI, uma licença só é considerada \textit{open source} se cumprir com a \textit{Open Source Definition}(OSD). Esta definição inclui uma série de critérios projetados para proteger a liberdade do
utilizador e fomentar a inovação. Por exemplo, uma licença \textit{open source} deve permitir redistribuições livres do software, acesso ao código-fonte e criação de obras derivadas.
O processo de aprovação de uma licença pela OSI é um pilar essencial para garantir que estas normas sejam mantidas. Através de um processo de revisão pública, a comunidade open source pode dar opiniões 
sobre novas licenças propostas para garantir que elas estejam alinhadas com os padrões estabelecidos. Este processo não apenas protege os direitos dos utilizadores e desenvolvedores, mas também mantém um 
padrão uniforme que facilita a colaboração e a partilha de tecnologia entre projetos e organizações.


\textbf{Tipos de licenças \textit{Open Source}}
Existem dois tipos principais de licenças open source: \textbf{\textit{copyleft}} e \textbf{permissivas}. As licenças \textit{copyleft}, como a \textbf{\textit{GNU General Public License}}, exigem que quaisquer versões
modificadas do software também sejam distribuídas com a mesma licença open source. Isso garante que as liberdades concedidas pela licença original sejam mantidas em todas as versões derivadas do software. 
Por outro lado, as licenças permissivas, como a licença \textbf{MIT} e a licença \textbf{BSD}, são menos restritivas, permitindo que o software seja integrado em projetos proprietários. 
Essas licenças ainda garantem liberdades fundamentais, mas não exigem que as obras derivadas sejam distribuídas sob os mesmos termos \textit{open source}.


\textbf{Impacto do licenciamento \textit{Open Source}}
O impacto do licenciamento \textit{open source} é profundo e abrangente. Este, permite que empresas, desde \textit{startups} até grandes organizações, inovem e construam sobre o trabalho existente sem as 
restrições de licenças de software proprietário.
Este ambiente de inovação aberta tem levado ao desenvolvimento de tecnologias significativas em campos como servidores web, smartphones, automação empresarial, computação em cloud e a economia partilhada. 
O licenciamento \textit{open source} apoia a inovação contínua e a disseminação rápida de tecnologias emergentes, beneficiando tanto os desenvolvedores individuais quanto a indústria tecnológica em larga escala.
