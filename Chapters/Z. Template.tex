\section{This is a section} \pagenumbering{roman}
\subsection{This is a subsection}

\subsubsection{This is a subsubsection}
This section contains some templates that can be used to create a uniform style within the document. It also shows of the overall formatting of the template, created using the predefined styles from the \texttt{settings.tex} file.

\subsection{General formatting}
Firstly, the document uses the font mlmodern, using no indent for new paragraphs and commonly uses the color \textcolor{Tue-red}{Tue-red} (the color of the TU/e logo) in its formatting. It uses the \texttt{fancyhdr} package for its headers and footers, using the TU/e logo and report title as the header and the page number as the footer. The template uses custom section, subsection and subsubsection formatting making use of the \texttt{titlesec} package.\\
The \texttt{hyperref} package is responsible for highlighting and formatting references like figures and tables. For example \cref{table: style 1} or \cref{fig: three images}. It also works for citations \cite{texbook}. Note how figure numbers are numbered according to the format \texttt{<chapter number>.<figure number>}.\\

Bullet lists are also changed globally, for a maximum of 3 levels:

\begin{itemize}
    \item Item 1
    \item Item 2
    \begin{itemize}
        \item subitem 1
        \begin{itemize}
            \item subsubitem 1
            \item subsubitem 2
        \end{itemize}
    \end{itemize}
    \item Item 3
\end{itemize}

Similarly numbered lists are also changed document wide:

\begin{enumerate}
    \item Item 1
    \item Item 2
    \begin{enumerate}
        \item subitem 1
        \begin{enumerate}
            \item subsubitem 1
            \item subsubitem 2
        \end{enumerate}
    \end{enumerate}
    \item Item 3
\end{enumerate}

\newpage

\subsection{Tables and figures}
The following table, \cref{table: style 1}, shows a possible format for tables in this document. Alternatively, one can also use the black and white version of this, shown in \cref{table: style 2}. Note that caption labels are in the format \textbf{\textcolor{Tue-red}{Table x.y:} }
\begin{table}[ht]
\rowcolors{2}{Tue-red!10}{white}
\centering
\caption{A table without vertical lines.}
\begin{tabular}[t]{ccccc}
\toprule
\color{Tue-red}\textbf{Column 1}&\color{Tue-red}\textbf{Column 2}&\color{Tue-red}\textbf{Column 3}&\color{Tue-red}\textbf{Column 4}&\color{Tue-red}\textbf{Column 5}\\
\midrule
Entry 1&1&2&3&4\\
Entry 2&1&2&3&4\\
Entry 3&1&2&3&4\\
Entry 4&1&2&3&4\\
\bottomrule
\end{tabular}
\label{table: style 1}
\end{table}

\begin{table}[ht]
\rowcolors{2}{gray!10}{white}
\centering
\caption{A table without vertical lines.}
\begin{tabular}[t]{ccccc}
\toprule
\textbf{Column 1}&\textbf{Column 2}&\textbf{Column 3}&\textbf{Column 4}&\textbf{Column 5}\\
\midrule
Entry 1&1&2&3&4\\
Entry 2&1&2&3&4\\
Entry 3&1&2&3&4\\
Entry 4&1&2&3&4\\
\bottomrule
\end{tabular}
\label{table: style 2}
\end{table}

For normal, single image figures, the standard \texttt{\textbackslash begin\{figure\}} environment can be used. For multi-image figures, one could use either the \texttt{\textbackslash begin\{subfigure\}} environment to get a main caption with 3 subcaptions like \cref{fig: three images} or the \texttt{\textbackslash begin\{minipage\}} environment to get 3 independent captions like \cref{fig: style 2 image a} - \ref{fig: style 2 image c}

\begin{figure}[H]
     \centering
     \begin{subfigure}[b]{0.3\textwidth}
         \centering
         \includegraphics[width=\textwidth]{example-image-a}
         \caption{image a}
         \label{fig: style 1 image a}
     \end{subfigure}
     \hfill
     \begin{subfigure}[b]{0.3\textwidth}
         \centering
         \includegraphics[width=\textwidth]{example-image-b}
         \caption{image b}
         \label{fig: style 1 image b}
     \end{subfigure}
     \hfill
     \begin{subfigure}[b]{0.3\textwidth}
         \centering
         \includegraphics[width=\textwidth]{example-image-c}
         \caption{image c}
         \label{fig: style 1 image c}
     \end{subfigure}
        \caption{Three images}
        \label{fig: three images}
\end{figure}

\begin{figure}[H]
\centering
\begin{minipage}{0.3\textwidth}
  \centering
  \includegraphics[width=\textwidth]{example-image-a}
  \captionof{figure}{image a}
  \label{fig: style 2 image a}
\end{minipage}
\hfill
\begin{minipage}{0.3\textwidth}
  \centering
  \includegraphics[width=\textwidth]{example-image-b}
  \captionof{figure}{image b}
  \label{fig: style 2 image b}
\end{minipage}
\hfill
\begin{minipage}{0.3\textwidth}
  \centering
  \includegraphics[width=\textwidth]{example-image-c}
  \captionof{figure}{image c}
  \label{fig: style 2 image c}
\end{minipage}
\end{figure}