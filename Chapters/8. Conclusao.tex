\section{Conclusão} \label{section: introduction}
Nos últimos anos, o panorama da escolha de um Sistema Operativo (SO) tem sido marcado pela convivência entre soluções open source e proprietárias. Essa dualidade não apenas influencia a seleção de um usuário individual, mas também afeta os ambientes empresariais e governamentais.

O presente trabalho de investigação tem como objetivo explorar essa tendência, bem como destacar a importância do Kernel em um SO, abordar a questão da segurança em software open source e discutir os diferentes tipos de licenciamentos. Nesse contexto, é crucial compreender as vantagens e desvantagens ao optar por soluções de software open source ou proprietário. A decisão vai além das preferências individuais, pois impacta diretamente a eficiência, segurança e os custos associados.

Ao analisar essa dualidade, é essencial não apenas considerar as características técnicas dos SO, mas também avaliar as implicações legais, como licenciamento e conformidade.

A segurança da informação surge como um tema crucial nessa escolha. Muitos consideram os SO open source mais transparentes e suscetíveis a auditorias, porém potencialmente menos seguros em comparação aos SO proprietários. Essa preocupação com a segurança não se limita aos SO, mas se estende a todo tipo de software em geral.

O trabalho seguirá um índice estabelecido, abordando os seguintes tópicos: Introdução, Sistemas Operativos open source e proprietários, Kernel e sua importância, exemplos de utilização do Linux, licenciamento open source, segurança em software open source, aquisição e uso de software open source em ambientes empresariais, conclusão e referências.